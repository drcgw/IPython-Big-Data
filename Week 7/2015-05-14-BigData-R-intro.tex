% Big Data Analytics
% Lecture on R
% C. G. Wilson
% May 13th, 2015

\documentclass{beamer}

\usetheme{Dresden}
\usecolortheme{beaver}
%%\setbeamertemplate{navigation symbols}{}
\setbeamertemplate{section in toc}[sections numbered]

% Brackets below are for short titles
\title[]{An Introduction to R}
%\subtitle[]{}
\author[]{Christopher G. Wilson, Ph.D.}
\date{\today}
\institute[LLU]{Loma Linda University\\
Dept. of Pediatrics and Center for Perinatal Biology}

\beamersetuncovermixins{\opaqueness<1>{25}}{\opaqueness<2->{15}}

\begin{document}

\begin{frame}
\titlepage
\end{frame}

\begin{frame}
\frametitle{Outline}
\tableofcontents
\end{frame}

\section[]{Project 1 due today}
\begin{frame}{Project 1}
Please make sure that the IPython notebook for Project 1 is 
in the class GitHub repository.
\end{frame}

\section[]{Why use R?}
\begin{frame}{Why should you use R}
\begin{itemize}
\item Complete statistical environment and programming language 
\item Efficient functions and data structures for data analysis 
\item Powerful graphics
\item Access to fast growing number of analysis packages
\item Most widely used language in bioinformatics
\item Is standard for data mining and biostatistical analysis 
\item Technical advantages include free, open-source, multi-platform
\end{itemize} 
\end{frame}

\begin{frame}{Books and Documentation}
\begin{itemize}
\item simpleR - Using R for Introductory Statistics (John Verzani, 2004) 
\item Bioinformatics and Computational Biology Solutions Using R and Bioconductor (Gentleman et al., 2005)
\item Some free tutorials/books are available at: \texttt{http://www.statmethods.net/about/books.html}
\item Extensive documentation available at \texttt{http://www.r-project.org/}
\end{itemize} 
\end{frame}

\begin{frame}{Package Depositories}
There are many contributed packages that make R extremely versatile
\begin{itemize}
\item CRAN ($>3500$ packages) general data analysis 
\item Bioconductor ($>700$ packages) bioscience data analysis 
\item Omegahat ($>30$ packages) programming interfaces
\end{itemize} 
\end{frame}

\section[]{R and supporting software installation}
\begin{frame}{R Installation}
\begin{itemize}
\item Install \emph{R} for your operating system from: \texttt{http://cran.r-project.org/mirrors.html} 
\item Install \emph{R-Studio} from: \texttt{http://www.rstudio.com/ide/download} 
\end{itemize}
\end{frame}

\begin{frame}{Installation of CRAN Packages}
From the command prompt, install packages:\\
\texttt{$>$ install.packages(c("pkg1", "pkg2"))}\\
\texttt{$>$ install.packages("pkg.zip", repos=NULL)}
\end{frame}

\begin{frame}{Installation of BioConductor Packages}
From the command prompt, install packages:\\
\texttt{$>$ source("http://www.bioconductor.org/biocLite.R")}\\
\texttt{$>$ library(BiocInstaller)}\\
\texttt{$>$ BiocVersion()}\\
\texttt{$>$ bioclite()}\\
\texttt{$>$ bioclite(c("pkg1", "pkg2"))}
\end{frame}

\section[]{Working within R}
\begin{frame}{Startup and Closing R}
\begin{itemize}
\item \textbf{Starting R:} The R GUI versions, including RStudio, under Windows and Mac OS X can be opened by double-clicking their icons. Alternatively, one can start it by typing `R' in a terminal (default under Linux)
\item \textbf{Closing R:} The R environment is controlled by hidden files in the startup directory: .RData, .Rhistory and .Rprofile (optional).
\item \textbf{Close R by typing:} \texttt{q()}
\end{itemize}
\end{frame}

\begin{frame}{Getting Around}
Create an object with the assignment operator $<- (or =)$\\ 
$> object <- ...$\\
List objects in current R session\\
$> ls()$\\
Return content of current working directory\\
$> dir()$\\
Return path of current working directory\\
$> getwd()$\\
Change current working directory\\
$> setwd("/home/user")$
\end{frame}

\begin{frame}{Basic R Syntax}
General R command syntax\\
$> object <-  function\_name(arguments)$\\
$> object <-  object[arguments]$\\
Finding help\\
$> ?function\_name$\\
Load a library\\
$> library("my_library")$\\
Lists all functions defined by a library\\
$> library(help="my\_library")$\\
Load library manual (PDF file)\\
$> vignette("my\_library")$
\end{frame}

\begin{frame}{Executing R Scripts}
\textbf{Execute an R script from within R:}\\

$> source("my\_script.R")$\\
\vspace{0.50cm}

\textbf{Execute an R script from command-line:}\\

\$ $Rscript~my\_script.R$\\
\$ $R~CMD~BATCH~my\_script.R$\\
\$ $R~$-~-$slave~<~my\_script.R$
\end{frame}

\begin{frame}{Reading and Writing External Data}
\textbf{Import data from tabular files into R:}\\
$> myDF <- read.delim("myData.xls", sep=``\backslash{t}'')$\\
%\vspace{0.25cm}
\textbf{Export data from R to tabular files:}\\
$> write.table(myDF, file=``myfile.xls'', sep=``\backslash{t}'', quote=FALSE, col.names=NA)$\\
\vspace{0.25cm}
\textbf{Copy and paste (e.g. from Excel) into R:}\\
\# On Windows/Linux systems:\\
$> read.delim(``clipboard'')$\\
\# On Mac OS X systems:\\
$> read.delim(pipe(``pbpaste''))$
\end{frame}

\begin{frame}
\begin{center}
\Large{For the remainder of class: Install \emph{R} and \emph{R-Studio} and work through
Holly Harlin's R tutorial}
 \end{center}
\end{frame}
%\vspace{0.25cm}
%\textbf{Copy and paste from R into Excel or other programs:}
%\begin{verbatim}
%> ## On Windows/Linux systems:
%> write.table(iris, "clipboard", sep="\backslash{t}", col.names=NA, quote=F) > ## On Mac OS X systems:
%> zz <- pipe('pbcopy', 'w')
%> write.table(iris, zz, sep="\backslash{t}", col.names=NA, quote=F)
%> close(zz)
%\end{verbatim}


\end{document}